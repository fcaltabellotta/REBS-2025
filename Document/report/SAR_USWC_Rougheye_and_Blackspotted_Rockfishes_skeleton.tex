% Options for packages loaded elsewhere
\PassOptionsToPackage{unicode}{hyperref}
\PassOptionsToPackage{hyphens}{url}
\PassOptionsToPackage{dvipsnames,svgnames,x11names}{xcolor}
%
\documentclass[
]{scrartcl}

\usepackage{amsmath,amssymb}
\usepackage{iftex}
\ifPDFTeX
  \usepackage[T1]{fontenc}
  \usepackage[utf8]{inputenc}
  \usepackage{textcomp} % provide euro and other symbols
\else % if luatex or xetex
  \usepackage{unicode-math}
  \defaultfontfeatures{Scale=MatchLowercase}
  \defaultfontfeatures[\rmfamily]{Ligatures=TeX,Scale=1}
\fi
\usepackage{lmodern}
\ifPDFTeX\else  
    % xetex/luatex font selection
\fi
% Use upquote if available, for straight quotes in verbatim environments
\IfFileExists{upquote.sty}{\usepackage{upquote}}{}
\IfFileExists{microtype.sty}{% use microtype if available
  \usepackage[]{microtype}
  \UseMicrotypeSet[protrusion]{basicmath} % disable protrusion for tt fonts
}{}
\makeatletter
\@ifundefined{KOMAClassName}{% if non-KOMA class
  \IfFileExists{parskip.sty}{%
    \usepackage{parskip}
  }{% else
    \setlength{\parindent}{0pt}
    \setlength{\parskip}{6pt plus 2pt minus 1pt}}
}{% if KOMA class
  \KOMAoptions{parskip=half}}
\makeatother
\usepackage{xcolor}
\setlength{\emergencystretch}{3em} % prevent overfull lines
\setcounter{secnumdepth}{5}
% Make \paragraph and \subparagraph free-standing
\makeatletter
\ifx\paragraph\undefined\else
  \let\oldparagraph\paragraph
  \renewcommand{\paragraph}{
    \@ifstar
      \xxxParagraphStar
      \xxxParagraphNoStar
  }
  \newcommand{\xxxParagraphStar}[1]{\oldparagraph*{#1}\mbox{}}
  \newcommand{\xxxParagraphNoStar}[1]{\oldparagraph{#1}\mbox{}}
\fi
\ifx\subparagraph\undefined\else
  \let\oldsubparagraph\subparagraph
  \renewcommand{\subparagraph}{
    \@ifstar
      \xxxSubParagraphStar
      \xxxSubParagraphNoStar
  }
  \newcommand{\xxxSubParagraphStar}[1]{\oldsubparagraph*{#1}\mbox{}}
  \newcommand{\xxxSubParagraphNoStar}[1]{\oldsubparagraph{#1}\mbox{}}
\fi
\makeatother


\providecommand{\tightlist}{%
  \setlength{\itemsep}{0pt}\setlength{\parskip}{0pt}}\usepackage{longtable,booktabs,array}
\usepackage{calc} % for calculating minipage widths
% Correct order of tables after \paragraph or \subparagraph
\usepackage{etoolbox}
\makeatletter
\patchcmd\longtable{\par}{\if@noskipsec\mbox{}\fi\par}{}{}
\makeatother
% Allow footnotes in longtable head/foot
\IfFileExists{footnotehyper.sty}{\usepackage{footnotehyper}}{\usepackage{footnote}}
\makesavenoteenv{longtable}
\usepackage{graphicx}
\makeatletter
\def\maxwidth{\ifdim\Gin@nat@width>\linewidth\linewidth\else\Gin@nat@width\fi}
\def\maxheight{\ifdim\Gin@nat@height>\textheight\textheight\else\Gin@nat@height\fi}
\makeatother
% Scale images if necessary, so that they will not overflow the page
% margins by default, and it is still possible to overwrite the defaults
% using explicit options in \includegraphics[width, height, ...]{}
\setkeys{Gin}{width=\maxwidth,height=\maxheight,keepaspectratio}
% Set default figure placement to htbp
\makeatletter
\def\fps@figure{htbp}
\makeatother
% definitions for citeproc citations
\NewDocumentCommand\citeproctext{}{}
\NewDocumentCommand\citeproc{mm}{%
  \begingroup\def\citeproctext{#2}\cite{#1}\endgroup}
\makeatletter
 % allow citations to break across lines
 \let\@cite@ofmt\@firstofone
 % avoid brackets around text for \cite:
 \def\@biblabel#1{}
 \def\@cite#1#2{{#1\if@tempswa , #2\fi}}
\makeatother
\newlength{\cslhangindent}
\setlength{\cslhangindent}{1.5em}
\newlength{\csllabelwidth}
\setlength{\csllabelwidth}{3em}
\newenvironment{CSLReferences}[2] % #1 hanging-indent, #2 entry-spacing
 {\begin{list}{}{%
  \setlength{\itemindent}{0pt}
  \setlength{\leftmargin}{0pt}
  \setlength{\parsep}{0pt}
  % turn on hanging indent if param 1 is 1
  \ifodd #1
   \setlength{\leftmargin}{\cslhangindent}
   \setlength{\itemindent}{-1\cslhangindent}
  \fi
  % set entry spacing
  \setlength{\itemsep}{#2\baselineskip}}}
 {\end{list}}
\usepackage{calc}
\newcommand{\CSLBlock}[1]{\hfill\break\parbox[t]{\linewidth}{\strut\ignorespaces#1\strut}}
\newcommand{\CSLLeftMargin}[1]{\parbox[t]{\csllabelwidth}{\strut#1\strut}}
\newcommand{\CSLRightInline}[1]{\parbox[t]{\linewidth - \csllabelwidth}{\strut#1\strut}}
\newcommand{\CSLIndent}[1]{\hspace{\cslhangindent}#1}

\usepackage{hyphenat}
\usepackage{graphicx}
% and their extensions so you won't have to specify these with
 % every instance of \includegraphics
 \usepackage{pdfcomment}
\DeclareGraphicsExtensions{.pdf,.jpeg,.png}
\usepackage{wallpaper} % for the background image on title page
\usepackage{geometry}
% set font

% added by Ross
% % set font - - depends upon the driver
% \ifPDFTeX
%  %% only want this in body section headings and ToC, using \sf
%  \def\sfdefault{phv}% Helvetica instead of its clone Arial
%  \renewcommand{\sectfont}{\normalcolor
%   \def\bfdefault{bc}% bold condensed; i.e., narrow
%   \maybesffamily \bfseries }%% uses uhvb8ac
% % \def\sfdefault{lmss}% Latin Modern replaces Arial
% % \renewcommand{\sectfont}{\normalcolor
%  % \fontseries{sbc}\fontfamily{lmss}\selectfont }%% uses lmssdc10
% \else

\usepackage{fontspec}
\setsansfont[Ligatures=TeX]{Arial Narrow}

% added by Ross
%\fi
%\usepackage[scaled=0.9]{helvet}% needed later to replace Arial Narrow
\usepackage[headsepline=0.005pt:,footsepline=0.005pt:,plainfootsepline,automark]{scrlayer-scrpage}
\clearpairofpagestyles
\ohead[]{\headmark} \cofoot[\pagemark]{\pagemark}
\lohead{Rougheye and Blackspotted Rockfishes assessment 2025}
\ModifyLayer[addvoffset=-.6ex]{scrheadings.foot.above.line}
\ModifyLayer[addvoffset=-.6ex]{plain.scrheadings.foot.above.line}
\setkomafont{pageheadfoot}{\small}

\usepackage{booktabs}
\usepackage{longtable}
\usepackage{array}
\usepackage{multirow}
\usepackage{wrapfig}
\usepackage{float}
\usepackage{colortbl}
\usepackage{pdflscape}
\usepackage{tabu}
\usepackage{threeparttable}
\usepackage{threeparttablex}
\usepackage[normalem]{ulem}
\usepackage{makecell}
\usepackage{xcolor}
\makeatletter
\@ifpackageloaded{caption}{}{\usepackage{caption}}
\AtBeginDocument{%
\ifdefined\contentsname
  \renewcommand*\contentsname{Table of contents}
\else
  \newcommand\contentsname{Table of contents}
\fi
\ifdefined\listfigurename
  \renewcommand*\listfigurename{List of Figures}
\else
  \newcommand\listfigurename{List of Figures}
\fi
\ifdefined\listtablename
  \renewcommand*\listtablename{List of Tables}
\else
  \newcommand\listtablename{List of Tables}
\fi
\ifdefined\figurename
  \renewcommand*\figurename{Figure}
\else
  \newcommand\figurename{Figure}
\fi
\ifdefined\tablename
  \renewcommand*\tablename{Table}
\else
  \newcommand\tablename{Table}
\fi
}
\@ifpackageloaded{float}{}{\usepackage{float}}
\floatstyle{ruled}
\@ifundefined{c@chapter}{\newfloat{codelisting}{h}{lop}}{\newfloat{codelisting}{h}{lop}[chapter]}
\floatname{codelisting}{Listing}
\newcommand*\listoflistings{\listof{codelisting}{List of Listings}}
\makeatother
\makeatletter
\makeatother
\makeatletter
\@ifpackageloaded{caption}{}{\usepackage{caption}}
\@ifpackageloaded{subcaption}{}{\usepackage{subcaption}}
\makeatother

\ifLuaTeX
\usepackage[bidi=basic]{babel}
\else
\usepackage[bidi=default]{babel}
\fi
\babelprovide[main,import]{english}
% get rid of language-specific shorthands (see #6817):
\let\LanguageShortHands\languageshorthands
\def\languageshorthands#1{}
\ifLuaTeX
  \usepackage{selnolig}  % disable illegal ligatures
\fi
\usepackage{bookmark}

\IfFileExists{xurl.sty}{\usepackage{xurl}}{} % add URL line breaks if available
\urlstyle{same} % disable monospaced font for URLs
\hypersetup{
  pdftitle={Status of the Rougheye and Blackspotted Rockfishes stock off the U.S. West Coast in 2025},
  pdfauthor={Jason M. Cope; Vladlena Gertseva; R. Claire Rosmond; Alison D. Whitman},
  pdflang={en},
  colorlinks=true,
  linkcolor={blue},
  filecolor={Maroon},
  citecolor={Blue},
  urlcolor={Blue},
  pdfcreator={LaTeX via pandoc}}


\title{Status of the Rougheye and Blackspotted Rockfishes stock off the
U.S. West Coast in 2025}
\author{Jason M. Cope \and Vladlena Gertseva \and R. Claire
Rosmond \and Alison D. Whitman}
\date{2025-04-14}

\begin{document}
  \begin{titlepage}
  % This is a combination of Pandoc templating and LaTeX
  % Pandoc templating https://pandoc.org/MANUAL.html#templates
  % See the README for help

  \newgeometry{top=2in,bottom=1in,right=1in,left=1in}
  \begin{minipage}[b][\textheight][s]{\textwidth}
  % Ross would've subbed lines 6, 8 with these lines:
  %\newgeometry{top=2in,bottom=1in,right=1in,left=1in}%
  %\noindent  %\tracingall
  %\begin{minipage}[b][\textheight][s]{.975\textwidth}%% RRM: avoid Overfull box


  \raggedright

  % \includegraphics[width=2cm]{NOAA_Transparent_Logo.png}

  % background image


  % Title and subtitle
  {\huge\bfseries\nohyphens{Status of the Rougheye and Blackspotted
  Rockfishes stock off the U.S. West Coast in 2025}}\\[1\baselineskip]
  % Ross would change the end of the above line to the following because \par must come before the group closes and line-depth reverts.
  % }\par}%\\[1\baselineskip]



  \vspace{1\baselineskip}
  % Ross would change this to 2\baselineskip

  %%%%%% Cover image

  \vspace{1\baselineskip}

  % Authors
  % This hairy bit of code is just to get "and" between the last 2
  % authors. See below if you don't need that
   {\large{Jason M. Cope}}{\textsuperscript{1}}%
  %
  ,
   {\large{Vladlena Gertseva}}{\textsuperscript{1}}%
  %
  ,
   {\large{R. Claire Rosmond}}{\textsuperscript{2}}%
  %
  %
  { and \large{Alison D. Whitman}}%
  {\textsuperscript{3}}%
  %


  % This is how to do it if you don't need the "and"

  %%%%%% Affiliations
  \vspace{2\baselineskip}

  \hangindent=1em
  \hangafter=1
  % Ross would change the above line to:
  % \hangafter=1\relax
  %
  {1}.~{NOAA Fisheries Northwest Fisheries Science Center}%
  %
  %
  % Ross recommends putting address on one line
  , %
  {2725 Montlake Boulevard East}%
  %
  \par\hangindent=1em\hangafter=1%
  %
  {2}.~{NOAA Fisheries Northwest Fisheries Science Center}%
  %
  %
  % Ross recommends putting address on one line
  , %
  {2032 SE Osu Drive}%
  %
  \par\hangindent=1em\hangafter=1%
  %
  {3}.~{Oregon Department of Fish and Wildlife}%
  %
  %
  % Ross recommends putting address on one line
  , %
  {2040 Southeast Marine Science Drive}%
  %


  %%%%%% Correspondence
  \vspace{1\baselineskip}


  %use \vfill instead to get the space to fill flexibly
  %\vspace{0.25\textheight} % Whitespace between the title block and the publisher

  \vfill


  % Whitespace between the title block and the tagline
  \vspace{1\baselineskip}

  %%%%%% Tagline at bottom
  % Ross says the tagline below could also be centered
  \includegraphics[alt={},width=2cm]{support_files/us_doc_logo.png}\newline % empty curly brackets without alt text is suitable for this logo because it's purely decorative/an "artifact"
  U.S. Department of Commerce\newline
  National Oceanic and Atmospheric Administration\newline
  National Marine Fisheries Service\newline
  Northwest Fisheries Science Center\newline

  \end{minipage}
  \restoregeometry
  \end{titlepage}

\renewcommand*\contentsname{Table of contents}
{
\hypersetup{linkcolor=}
\setcounter{tocdepth}{3}
\tableofcontents
}
\listoffigures
\listoftables

\newpage{}

Please cite this publication as:

Cope, J.M., V. Gertseva, R.C. Rosemond, F. Caltabellotta, A.D. Whitman.
Status of the Rougheye and Blackspotted Rockfishes stock off the U.S.
West Coast in 2025.2025. Prepared by {[}COMMITTEE{]}. {[}XX{]} p.

\newpage{}

\#\texttt{\{r,\ results=\textquotesingle{}asis\textquotesingle{}\}\ \#\#\textbar{}\ label:\ \textquotesingle{}load\_tables\textquotesingle{}\ \#\#\textbar{}\ eval:\ true\ \#\#\textbar{}\ echo:\ false\ \#\#\textbar{}\ warning:\ false\ \#a\ \textless{}-\ knitr::knit\_child(\textquotesingle{}002\_load\_tables.qmd\textquotesingle{},\ quiet\ =\ TRUE)\ \#cat(a,\ sep\ =\ \textquotesingle{}\textbackslash{}n\textquotesingle{})\ \#}

\section*{Disclaimer}\label{disclaimer}
\addcontentsline{toc}{section}{Disclaimer}

These materials do not constitute a formal publication and are for
information only. They are in a pre-review, pre-decisional state and
should not be formally cited or reproduced. They are to be considered
provisional and do not represent any determination or policy of NOAA or
the Department of Commerce.

\newpage{}

\subsection{Executive Summary}\label{executive-summary}

\subsubsection{Stock Description}\label{stock-description}

\subsubsection{Catches}\label{catches}

\subsubsection{Data and Assessments}\label{data-and-assessments}

\subsubsection{Stock Output and
Dynamics}\label{stock-output-and-dynamics}

\subsubsection{Recruitment}\label{recruitment}

\subsubsection{Exploitation Status}\label{exploitation-status}

\subsubsection{Ecosysystem
Consideration}\label{ecosysystem-consideration}

\subsubsection{Reference Points}\label{reference-points}

\subsubsection{Management Performance}\label{management-performance}

\subsubsection{Evaluation of Scientific
Uncertainty}\label{evaluation-of-scientific-uncertainty}

\subsubsection{Harvest Projections and Decision
Tables}\label{harvest-projections-and-decision-tables}

\subsubsection{Unresolved Problems and Major
Uncertainties}\label{unresolved-problems-and-major-uncertainties}

\subsubsection{Research and Data Needs}\label{research-and-data-needs}

\newpage{}

\section{Introduction}\label{introduction}

This document presents the stock assessment for the Rougheye
(\emph{Sebastes aleutianus}) and Blackspotted (\emph{Sebastes
melanostictus}) rockfishes, two species that form one management
complex. This report is for the year 2025 in state and federal waters
from California to Washington State, excluding consideration of the
Puget Sound and Salish Sea. It seeks to use available catch, biological
compositions in the for of lengths and ages, and potential indices of
abundance.

\subsection{Stock Structure}\label{stock-structure}

Rougheye rockfish were first described in 1811 as \emph{Perca
variabilis} by German zoologist Peter Simon Pallas (Jordan and Evermann
1898), and assigned to various taxa at least 15 times since (Love et
al.~2002). Some descriptions noted both light and dark color morphs,
which, along with possible confusion with several morphologically
similar co-occurring species (e.g., S. borealis and S. melanostomus)
have contributed to the persistent ambiguity in formal descriptions of
Rougheye Rockfish (Orr and Hawkins 2008). The first genetic studies
conducted in the late 1960s and early 1970s (e.g., Tsuyuki et al.~1968,
Tsuyuki and Westrheim 1970) observed diversity suggestive of two genetic
types within specimens identified as Rougheye Rockfish. Allozyme studies
conducted over the next two decades (e.g., Seeb 1986, Hawkins et
al.~1997, Hawkins et al.~2005) provided additional evidence suggesting
two separate genetic types within field-identified Rougheye Rockfish.
Genetic variation between the two types, supported by both nuclear and
mitochondrial DNA, was determined to be sufficiently conclusive to
separate two species: ``Type I'' and ``Type II'' Rougheye Rockfish
(Gharrett et al.~2005). Meristic and morphometric comparisons of the two
species suggested certain characters such as gill raker counts and
length, snout length, anal base length, and pectoral fin base were
significantly different, and in combination could reliably, though not
definitively, distinguish between the species (Gharrett et al.~2006).
The two separate species were formally re-described by Orr and Hawkins
(2008) with the Type II group retaining \emph{S. aleutianus} and the
common name Rougheye Rockfish. Blackspotted rockfish was proposed as the
common name for the Type I group along with the scientific name of
\emph{S. melanostictus}, re-establishing nomenclature from one of the
species complex's earlier descriptions (cf.~Matsubara 1934).

Though Rougheye and Blackspotted rockfishes are genetically distinct,
they remain difficult to visually distinguish, thus most data
historically and contemporaneously are only available for the
Rougheye/Blackspotted rockfish complex, not at the species level. They
both range from northern California up to and throughout Alaska. They
both greatly overlap in latitude and depth, and are generally considered
slope rockfish, with an otogentic shift from shallower to deeper, and
adults commonly found at 360 m. Rougheye seems to be proportionally more
abundant when survey samples are genetically identified, and
Blackspotted tend to be found, on average, deeper than Rougheye. They
can school and may segregate by size and age. While we treat these
species as one assessed stock from this point forward, we recognize and
are mindful of the above distinctions as we conduct our analyses.

There are at least two questions to think about when considering stock
structure for Rougheye and Blackspotted rockfishes.

\begin{enumerate}
\def\labelenumi{\arabic{enumi}.}
\item
  Rougheye and Blackspotted rockfishes are two different species-- can
  we separate them as two stocks and conduct separate assessments? These
  two species remain difficult to differentiate visually in the catch,
  thus remain reported as a species complex. Frey et al.~(in prep.)
  provided insight into the ability of the Northwest Fisheries Science
  Center West Coast Groundfish Bottom Trawl Survey biologists to
  identify the two species, with 90\% of genetically identified Rougheye
  rockfish being correctly identified in the field. When
  mis-identifications occured, it was usually a Blackspotted rockfish
  being mis-identified as a Rougheye rockfish. There were few
  mis-identifications when a fish was identified as a Blackspotted
  rockfish. While this is promising for potential future
  species-specific data coming from the survey, it does not alleviate
  the historical problem of separating fishery data into the two
  species. Frey et al.~(in prep.) therefore also considered whether
  ecological factors like depth or latitude could help separate samples
  by species. They found that both species occur within the range of
  this assessment's considered areas (California to Washington), and
  heavily spatially overlap. Interestingly, there seem to be relative
  hot spots for these species where one species is more common than the
  other, and in general, Rougheye rockfish seems to be more common than
  Blackspotted rockfish (however, Blackspotted rockfish may be the more
  common of the two in parts of Alaska). Overall, there seems to be
  little ability to separate current or historical fishery data reliably
  in order to separate these two species into two stocks, so we will
  maintain a species complex approach, though given absolute presence
  off the U.S. West coast, this may be considered more of a Rougheye
  than Blackspotted stock assessment. We also note that throughout the
  range of these stocks, all assessments have maintained a species
  complex approach.
\item
  Both species range into Canada and Alaska-- are they one stock? While
  genetics studies have focused mostly on identification of the two
  species, little is known about the population structure of either
  species. This assessment and the 2013 assessment represent the most
  southerly range of these species. Comparing the absolute abundance of
  the 2013 assessment to the most current estimates of the Alaskan
  stocks, the absolute number in this southerly range is much smaller
  than in the Gulf of Alaska (GOA), but higher than the Bering
  Sea/Aleutian Island (BSAI) stock (Figure~\ref{fig-SO_comp}). The two
  smaller stocks have similar trend of decline and stabilization,
  whereas the higher biomass GOA stock looks to have not dropped at all
  over the time period considered (Figure~\ref{fig-RSS_comp}).
\end{enumerate}

These two species may hybridize on occasion (Love 2011). These species
are closely related to shortraker rockfish (S. borealis) and are
sometimes difficult to distinguish from shortraker rockfish without
looking at the gill rakers.

\subsection{Life History Information}\label{life-history-information}

Rougheye and blackspotted rockfish share broad overlap in their depth
and geographic distributions from the Eastern Aleutian Islands along the
North American continental margin to southern Oregon, with blackspotted
rockfish's range extending east beyond the Aleutian chain to the Pacific
Coast of Japan (Gharrett et al.~2005, Hawkins et al.~2005, Orr and
Hawkins 2008). Both species are encountered at depths shallower than 100
m to at least 439 m, however, blackspotted rockfish tend to be more
prevalent in deeper waters (Hawkins et al.~2005, Orr and Hawkins 2008).
Genetic information is not available to provide positive species
identification in historical survey and landings information, but these
data indicate that density of the nominal rougheye rockfish complex
decreases sharply south of the Oregon-California border (42° N). Studies
suggest that rougheye rockfish account for a greater proportion of the
species complex along the coast of Washington and Oregon than in Alaskan
waters (Gharrett et al.~2005, Hawkins et al.~2005, Orr and Hawkins
2008). Recent discussions with port samplers in southern Oregon suggest
that both rougheye and blackspotted rockfish are encountered with some
regularity in the commercial trawl and fixed-gear landings in
Charleston, Port Orford, and Brookings, with blackspotted rockfish
composing approximately one third to one half of identified specimens
(C. Good and N. Wilsman, ODFW, pers. comm.).

The west coast of the U.S. is the southern portion of the range of
rougheye rockfish, and it is likely that the population north of the
U.S.-Canada border is not a separate stock. The connectivity of rougheye
populations throughout its range is unknown.

\subsection{Life History}\label{life-history}

\subsection{Ecosystem considerations}\label{ecosystem-considerations}

\subsection{Historical and Current Fishery
Information}\label{historical-and-current-fishery-information}

\subsection{Management History}\label{management-history}

\subsection{Management performance}\label{management-performance-1}

\subsection{Fisheries off Canada and
Alaska}\label{fisheries-off-canada-and-alaska}

\newpage{}

\section{Data}\label{data}

Data from a wide range of programs were available for possible inclusion
in the current assessment model. Descriptions of each data source
included in the model (Figure ) and sources that were explored but not
included in the base model are provided below. Data that were excluded
from the base model were excluded only after being explicitly explored
during the development of this stock assessment and found to be
inappropriate for use or had not changed since their past exploration
for previous Rougheye/Blackspotted rockfishes stock assessments when
they were not used.

\subsection{Fishery-dependent data}\label{fishery-dependent-data}

\subsubsection{Landings}\label{landings}

\paragraph{Trawl}\label{trawl}

\paragraph{Non-trawl}\label{non-trawl}

\paragraph{At-sea-hake fishery}\label{at-sea-hake-fishery}

\subsubsection{Discards}\label{discards}

\paragraph{Trawl}\label{trawl-1}

\paragraph{Non-trawl}\label{non-trawl-1}

\subsubsection{Biological data}\label{biological-data}

\paragraph{Length and Age Sample
Sizes}\label{length-and-age-sample-sizes}

\subparagraph{Multinomial Sample Sizes}\label{multinomial-sample-sizes}

Initial input values for the multinomial samples sizes determine the
relative weights applied in fitting the annual composition data within
the set of observations for each fishing fleet in the model. The initial
input values in this assessment were based on the following equation
developed by I. Stewart and S. Miller (NWFSC), and presented at the 2006
Stock Assessment Data and Modeling workshop. The input sample sizes for
all commercial data were calculated based on a combination of trips and
fish sampled:

\begin{centering}

Input effN = $N_{\text{trips}} + 0.138 * N_{\text{fish}}$ if $N_{\text{fish}}/N_{\text{trips}}$ is $<$ 44

Input effN = $7.06 * N_{\text{trips}}$ if $N_{\text{fish}}/N_{\text{trips}}$ is $\geq$ 44

\end{centering}

\paragraph{Trawl}\label{trawl-2}

\paragraph{Non-trawl}\label{non-trawl-2}

\paragraph{At-sea-hake fishery}\label{at-sea-hake-fishery-1}

\subsection{Fishery-independent data}\label{fishery-independent-data}

\subsubsection{Abundance indices}\label{abundance-indices}

\paragraph{NWFSC West Coast Groundfish Bottom Trawl
Survey}\label{nwfsc-west-coast-groundfish-bottom-trawl-survey}

\paragraph{NWFSC Slope Survey}\label{nwfsc-slope-survey}

\paragraph{AFSC Slope Survey}\label{afsc-slope-survey}

\paragraph{AFSC/NWFSC West Coast Triennial Shelf
Survey}\label{afscnwfsc-west-coast-triennial-shelf-survey}

\subsubsection{Biological data}\label{biological-data-1}

\paragraph{NWFSC West Coast Groundfish Bottom Trawl
Survey}\label{nwfsc-west-coast-groundfish-bottom-trawl-survey-1}

\paragraph{NWFSC Slope Survey}\label{nwfsc-slope-survey-1}

\paragraph{AFSC Slope Survey}\label{afsc-slope-survey-1}

\paragraph{AFSC/NWFSC West Coast Triennial Shelf
Survey}\label{afscnwfsc-west-coast-triennial-shelf-survey-1}

\subsection{Biological Parameters}\label{biological-parameters}

\subsubsection{Natural Mortality}\label{natural-mortality}

\begin{centering}

$M=\frac{5.4}{A_{\text{max}}}$

\end{centering}

\vspace{0.5cm}

where \(M\) is natural mortality and \({A_{\text{max}}}\) is the assumed
maximum age. The prior is defined as a lognormal distribution with mean
\(ln(5.4/A_{\text{max}})\) and standard error = 0.31.

\subsubsection{Age and Growth
Relationship}\label{age-and-growth-relationship}

\begin{centering}

Females $L_{\infty}$ =  cm; $k$ =  per year; $t_0$ = 

Males $L_{\infty}$ =  cm; $k$ =  per year; $t_0$ = 

\end{centering}

\subsubsection{Ageing Bias and
Precision}\label{ageing-bias-and-precision}

\subsubsection{Length-Weight
Relationship}\label{length-weight-relationship}

\subsubsection{Maturity}\label{maturity}

\subsubsection{Fecundity}\label{fecundity}

\subsubsection{Stock-Recruitment Function and
Compensation}\label{stock-recruitment-function-and-compensation}

\subsubsection{Sex Ratio}\label{sex-ratio}

No information on the sex ratio at birth was available so it was assumed
to be 50:50.

\subsection{Environmental and ecosystem
data}\label{environmental-and-ecosystem-data}

This stock assessment does not explicitly incorporate trophic
interactions, habitat factors or environmental factors into the
assessment model. More predation, diet and habitat work, and mechanistic
linkages to environmental conditions would be needed to incorporate
these elements into the stock assessment and should remain a priority.
McClure et al. (\textbf{mcclure\_vulnerability\_2023?}) report the
climate vulnerability for several west coast groundfishes, including
Rougheye/Blackspotted rockfishes. Rougheye/Blackspotted rockfishes
demonstrated both high biological sensitivity and high climate exposure
risk, to give it an overall high vulnerability score to climate change.
This result should also be considered with the fact that, like many
rockfishes, periods of low productivity is not unusual to
Rougheye/Blackspotted rockfishes and their extended longevity (though
admittedly this seems shorter than previously believed and should be
reconsidered) has historically allowed them to wait for advantageous
productivity periods. Stressors such as habitat degradation and climate
change could bring significant challenges to population sustainability.
Regardless, no environmental or ecosystem data are directly incorporated
into the stock assessment model.

\newpage{}

\subsection{Assessment}\label{assessment}

\subsubsection{History of Modeling
Approaches}\label{history-of-modeling-approaches}

\subsubsection{Most Recent STAR Panel
Recommendations}\label{most-recent-star-panel-recommendations}

\subsubsection{Response to SSC Groundfish Subcommittee
Recommendations}\label{response-to-ssc-groundfish-subcommittee-recommendations}

\subsection{Current Modelling
Platform}\label{current-modelling-platform}

Stock Synthesis version 3.30.22.1 was used as the statistical
catch-at-age modelling framework. This framework allows the integration
of a variety of data types and model specifications. The Stock
Assessmetn Continuum tool (https://github.com/shcaba/SS-DL-tool) was
used also used to explore model effciency, likelihood profiling,
retrospective analyses, and plotting sensitivities. The companion R
package r4ss (version 1.38.0) along with R version 4.2.2 were used to
investigate and plot model fits.

\subsubsection{Bridging the Assessment Model from Stock Synthesis 3.24
(2013) to 3.30
(2025)}\label{bridging-the-assessment-model-from-stock-synthesis-3.24-2013-to-3.30-2025}

Since several years have passed from the last assessment model, the
Stock Synthesis (SS3) modelling framework has undergone many changes.
While the specific changes in the model can be found in the model
\href{https://github.com/nmfs-stock-synthesis/stock-synthesis/blob/v3.30.19/Change_log_for_SS_3.30.xlsx?raw=true}{change
log}, here we simply update the model from the older 3.24O version to
the newer 3.30.22.1 version. The point here is to present any
differences in the model outputs when using the same information. This
was first done by migrating the data and parameter specifications from
the former files to the newer files. This migration was assisted using
the \href{https://github.com/shcaba/SS-DL-tool}{SS-DL tool}. Once the
old data was transferred to the SS 3.30.22.1 file, the model was run
allowing the same parameters estimation specification as in the 2013
model.

These model comparisons are adequate to move ahead using the newest
version of SS3 3.30.21 without expecting large differences in reference
models being due to versions of SS3.

\subsection{Model Structure, Evaluation, and
Specification}\label{model-structure-evaluation-and-specification}

\subsubsection{Fleet and Survey
Designations}\label{fleet-and-survey-designations}

The Washington model is structured to track several fleets and include
data from several surveys:

\begin{itemize}
\tightlist
\item
  Fleet 1: Commerical trawl fishery
\item
  Fleet 2: Commercial non-trawl (mostly jig) fishery
\item
  Fleet 3: Recreational boat fishery
\item
  Survey 1: Private boat\\
\item
  Survey 2: Charter
\item
  Survey 3: Tagging
\item
  Survey 4: Nearshore
\item
  Survey 5: OCNMS subadult-adult survey
\item
  Survey 6: OCNMS young-of-the-year survey
\end{itemize}

The specifications of the assessment are listed in Table
\ref{tab:model-structure}.

\subsection{Model Likelihood
Components}\label{model-likelihood-components}

There are five primary likelihood components for each assessment model:

\begin{enumerate}
\def\labelenumi{\arabic{enumi}.}
\tightlist
\item
  Fit to survey indices of abundance.
\item
  Fit to length composition samples.
\item
  Fit to age composition samples (all fit as conditional age-at-length).
\item
  Penalties on recruitment deviations (specified differently for each
  model).
\item
  Prior distribution penalties
\end{enumerate}

\subsection{Reference Model Exploration, Key Assumptions and
Specification}\label{reference-model-exploration-key-assumptions-and-specification}

The reference model for Washington Rougheye/Blackspotted rockfishes was
developed to balance parsimony and realism, and the goal was to estimate
a spawning output trajectory and relative stock status for the
population of Rougheye/Blackspotted rockfishes in state and federal
waters off Washington. The model contains many assumptions to achieve
parsimony and uses different data types and sources to estimate reality.
A series of investigative model runs were done to achieve the final base
model. Constructing integrated models (i.e., those fitting many data
types) takes considerable model exploration using different
configurations of the following treatments:

\begin{itemize}
\tightlist
\item
  Data types
\item
  Parameter treatments: which parameter can, cannot and do not need to
  be estimated
\item
  Phasing of parameter estimation
\item
  Data weighting
\item
  Exploration of local vs global minima (see Model Convergence and
  Acceptability section below)
\end{itemize}

The different biological data with and without the catch time series
(and no additional data weighting) were first included to obtain an
understanding of the signal of stock status coming from the data (Figure
). The length and age only models assume a constant catch over the
entire time series, while estimating the selectivity of each fleet.
Under this constraint, the lengths suggest a stock a bit lower than the
reference model, while the ages consider the stock is extremely
depleted. Putting the two data sources together produce an intermediate
stock status in the lower precautionary zone. Adding the catch time
series substantially changes the stock status trajectory, with length or
age only model above the reference stocks status. Combining the two came
out just under the reference model. Only one model includes recruitment
deviations, and demonstrates more dynamics behavior similar to that seen
when biological compositions are unweighted (see Model Specification
Sensitivities section ).

Stock scale was comparable once removal history was included, and
demonstrates a large sensitivity to the scale of the stock given the
data with no additional weighting included (Figure ).

Numerous exploratory models that included all data types and a variety
of model specifications were subsequently explored and too numerous to
fully report. In summary, the estimation of which life history
parameters to estimate and fix was liberally explored.

The following is a list of things that were explored, typically in
combination with one another

\begin{itemize}
\tightlist
\item
  Estimate or fix \(M\)
\item
  Estimate or fix any of the three growth parameter for each sex
\item
  Estimate or fix the stock-recruit relationship
\item
  Estimate or assume constant recruitment. If estimating recruitment,
  for what years?
\item
  Estimate or fix survey catchability for each survey
\item
  Estimate additional survey variance for which survey
\item
  Estimate or fix selectivity parameters
\item
  Logistic or dome-shaped selectivity?
\end{itemize}

After much consideration, it was determined that some parameters were
inestimable (\(M\), \(L_{min}\) for both sexes), some did not move much
for initial values and could be fixed (e.g., CV at length values, some
selectivity parameters), and others could be estimated (e.g.,
\(L_{\infty}\), \(k\), \(lnR_0\)). Estimation of \(L_min\) returned very
high estimates of \(L_{\infty}\) for both sexes, thus the \(L_{min}\)
value for both females and males was fixed to the external estimates. No
priors were used on any of the estimated parameters except female
\(L_{\infty}\) which used a normal prior and a standard deviation set a
bit higher from the external fit to the growth curve (0.2).
Length-at-maturity, fecundity-weight, and length-weight relationship,
steepness (\(h\)) and recruitment variance were all fixed.

The selectivity of all fisheries were estimated as logistic even if
dome-shaped selectivity was an option (and starting values begin at a
strong dome-shaped position). Constant selectivity was assumed for the
whole time period as there was no reason to suggest otherwise, and is
consistent with the previous stock assessment treatment.

The full list of estimate and fixed parameters are found in Table \}.

The biggest uncertainty was in the treatment of sex-specific \(M\), as
estimation came in very low for both sexes versus observed ages in the
population and the treatment in the last assessment. This parameter
affects both scale and status, and thus is a valuable parameter to
consider for characterizing model specification error and defining
states of nature. Both likelihood profiles and sensitivities explore the
influence of this parameter on derived model outputs.

General attributes of the reference model are that indices of abundance
are assumed to have lognormal measurement errors. Length compositions
and conditional age at length samples are all assumed to follow a
multinomial sampling distribution, where the sample size is fixed at the
input sample size calculated during compositional example, and where
this input sample size is subsequently reweighted to account for
additional sources of overdispersion (see below). Recruitment deviations
were also estimated are assumed to follow a lognormal distribution,
where the standard deviation of this distribution is tuned as explained
below.

Sensitivity scenarios and likelihood profiles (on \(lnR_0\), steepness,
and natural mortality) were used to explore uncertainty in the above
model specifications and are reported below.

\subsubsection{Data Weighting}\label{data-weighting}

The reference model allowed for the estimation of additional variance on
all surveys except the taggin and OCNMS adult survey, both of which
already had very high input variances. The ability to add variance to
indices allows the model to balance model fit to that data while
acknowledging that variances may be underestimated in the index
standardization. A sensitivity was run with no extra variance estimated,
as well as removal of the index data were explored.

Initial sample sizes for the commercial and recreational fleet length
and conditional age-at-length compositions were based on the number of
input effective samples sizes. The method of Francis (2011), equation
TA1.8, was then used to balance the length and conditional age-at-length
composition data among other inputs and likelihood components. The
Francis method treats mean length and age as indices, with effective
sample size defining the variance around the mean. If the variability
around the mean does not encompass model predictions, the data should be
down-weighted until predictions fit within the intervals. This method
accounts for correlation in the data (i.e., the multinomial
distribution), but can be sensitive to years that are outliers, as the
amount of down-weighting is applied to all years within a data source,
and are not year-specific. Sensitivities were performed examining
different data-weighting treatments: 1) the Dirichlet-Multinomial
approach (Thorson et al. 2017), 2) the McAllister-Ianelli Harmonic Mean
approach (McAllister and Ianelli 1997), or 3) no data-weighting of
lengths.

\subsubsection{Model Changes from the Last
Assessment}\label{model-changes-from-the-last-assessment}

Besides the additional of eight years of data and some changes in the
estimation of some parameters, the biggest changes to the 2015
assessment are:

\begin{itemize}
\tightlist
\item
  Change in the removal history, especially the trawl fishery 3A catches
  from Astoria.
\item
  Breaking the dockside survey into separate private and charter boat
  surveys. This allowed the ability to exclude years in the charter boat
  fishery that showed more effects from bag limits.
\item
  Addition of the nearshore survey, and both OCNMS surveys.
\end{itemize}

\newpage{}

\subsubsection{Reference Model Diagnostics and
Results}\label{reference-model-diagnostics-and-results}

\paragraph{Model Convergence and Acceptability}\label{model-convergence}

\subsubsection{Base Model Results}\label{base-model-results}

\paragraph{Fits to the Data}\label{fits-to-the-data}

\subparagraph{Lengths}\label{lengths}

\paragraph{Conditional Age at Length and Marginal
Ages}\label{conditional-age-at-length-and-marginal-ages}

\paragraph{Fits to Indices of
Abundance}\label{fits-to-indices-of-abundance}

\subsubsection{Reference Model Outputs}\label{reference-model-outputs}

\paragraph{Parameter Estimates}\label{parameter-estimates}

\paragraph{Population Trajectory}\label{population-trajectory}

\newpage{}

\subsection{Characterizing
uncertainty}\label{characterizing-uncertainty}

\subsubsection{Sensitivity Analyses}\label{sec-assmt-sens}

Sensitivity analyses were conducted to evaluate model sensitivity to
alternative data treatment and model specifications.

\paragraph{Data treatment
sensitivities}\label{data-treatment-sensitivities}

Data treatments explored were as follows:

\begin{itemize}
\tightlist
\item
  Treatment of abundance indiecs 1. 2015 dockside survey

  \begin{enumerate}
  \def\labelenumi{\arabic{enumi}.}
  \setcounter{enumi}{1}
  \tightlist
  \item
    2015 dockside survey, no extra variance estimated
  \item
    No extra variance on private boat index
  \end{enumerate}

  \begin{itemize}
  \tightlist
  \item
    Data weighting
  \end{itemize}

  \begin{enumerate}
  \def\labelenumi{\arabic{enumi}.}
  \setcounter{enumi}{10}
  \tightlist
  \item
    No data-weighting
  \item
    Dirichlet data-weighting
  \item
    McAllister-Ianelli data weighting
  \end{enumerate}
\item
  Other

  \begin{enumerate}
  \def\labelenumi{\arabic{enumi}.}
  \setcounter{enumi}{13}
  \tightlist
  \item
    2015 removal history
  \end{enumerate}
\end{itemize}

Likelihood values and estimates of key parameters and derived quantities
from each sensitivity are available in Table . Derived quantities
relative to the reference model are provided in Figure . Time series of
spawning output and relative spawning output are shown in Figures and .

\paragraph{Model Specification Sensitivities}\label{senstivities}

Model specifications looked at the estimation of individual and
combinations of life history parameters, the estimation of recruitment,
and the treatment of fecundity and selectivity. All scenarios match the
reference model specifications in all other aspects unless otherwise
stated.

\begin{itemize}
\tightlist
\item
  Life history estimation

  \begin{itemize}
  \tightlist
  \item
    Natural mortality (\(M\))

    \begin{enumerate}
    \def\labelenumi{\arabic{enumi}.}
    \tightlist
    \item
      Estimate \(M\)
    \item
      Lorenzen age varying \(M\)
    \item
      Use Oregon 2023 assessment sex-specific M values (females = 0.19;
      males = 0.17)
    \item
      Maintain sex ratio in age and length data (sex option 3) and
      estimate \(M\)
    \end{enumerate}
  \item
    Growth parameters

    \begin{enumerate}
    \def\labelenumi{\arabic{enumi}.}
    \setcounter{enumi}{5}
    \tightlist
    \item
      Fix all growth parameters to external values
    \item
      Fix all growth parameters to external values, estimate \(M\)
    \item
      Estimate \(L_min\)
    \item
      Fix \(t_0\) = 0
    \item
      Estimate \(CV_{young}\) and \(CV_{old}\)
    \end{enumerate}
  \item
    Reproductive Biology

    \begin{enumerate}
    \def\labelenumi{\arabic{enumi}.}
    \setcounter{enumi}{9}
    \tightlist
    \item
      Use biological maturity ogive
    \item
      Use functional maturity ogive
    \item
      Fecundity proportional to weight
    \end{enumerate}
  \item
    Recruitment estimation

    \begin{enumerate}
    \def\labelenumi{\arabic{enumi}.}
    \setcounter{enumi}{12}
    \tightlist
    \item
      No recruitment estimation
    \item
      Estimate recruitment for all years in the model
    \end{enumerate}
  \end{itemize}
\end{itemize}

Other

Likelihood values and estimates of key parameters and derived quantities
from each sensitivity are available in Table . Derived quantities
relative to the reference model are provided in Figure . Time series of
spawning output and relative spawning output are shown in Figures and .
None of the sensitivities indicated an overfished stock.

\subsubsection{Likelihood Profiles}\label{likelihood-profiles}

\subsubsection{Retrospective Analysis}\label{retrospective-analysis}

A five-year retrospective analysis was conducted by running the model
and sequentially removing one year of data up through minus 5 years.
Retrospective spawning output (Figure ) and relatives stock status
(Figure ) estimates show a generally consistent pattern in population
scale and trend, within the error of the reference model. All models
show the population increasing. This results in a stock status in the
precautionary zone over the 5 year consideration. The Mohn's rho
evaluation of the degree of retrospective pattern in given in Table and
shown in Figure . The relative error in the data peels are below
significant levels.

\subsubsection{Unresolved Problems and Major
Uncertainties}\label{unresolved-problems-and-major-uncertainties-1}

There are no major unresolved problems in the stock assessment, but
there are many sources of uncertainty. Natural mortality remains a large
source of uncertainty. The estimation of growth also required fixing
certain parameters, leading to an underestimation of uncertainty in the
model. The stock-recruit relationship is assumed to be a Beverton-Holt
relationship with a fixed steepness of 0.72. Large uncertainty was shown
if the nature of this relationship varies either deterministically or
over time. The full time series of recruitment deviations were not
informed, which creates some historical and contemporary uncertainty.
Likewise, all life history values are assumed constant, so any
time-varying issues that are directional could create more uncertainty.

\newpage{}

\section{Management}\label{management}

\subsection{Reference Points}\label{reference-points-1}

Reference points were based on the rockfish FMSY proxy
(\(\text{SPR}_{50\%}\)), target relative biomass (40\%), and estimated
selectivity and catch for each fleet (Table \ref{tab:ReferencePoints}).
The Rougheye/Blackspotted rockfishes population in Washington at the
start of 2023 is estimated to be just above the target biomass, and
fishing intensity during 2022 is estimated to be just below the fishing
intensity target (Figure \ref{fig:phase}). The yield values are lower
than the previous assessment for similar reference points due to updated
life history estimates and estimates of the total scale of the
population, despite the overall stock status being a bit higher. The
proxy MSY values of management quantities are by definition more
conservative compared to the estimated MSY and MSY relative to 40\% of
unfished spawning output because of the assumed steepness value.
Sustainable total yield, removals, using the proxy \(\text{SPR}_{50\%}\)
is 471 mt. The spawning output equivalent to 40\% of the unfished
spawning output (\(\text{SO}_{40\%}\)) calculated using the SPR target
(\(\text{SPR}_{50\%}\)) was 7,828 billions of eggs.

Recent removals since 2017 have been at or below the point estimate of
potential long-term yields calculated using an \(\text{SPR}_{50\%}\)
reference point (Figure \ref{fig:1-spr}), leading to a population that
has continued to increase over recent years with the assistance of above
average recruitment between 2003-2014, despite below average recruitment
starting in 2015. The equilibrium estimates of yield relative to biomass
based on a steepness value fixed at are provided in Figure
\ref{fig:yield}, where vertical dashed lines indicate the estimate of
fraction unfished at the start of 2027 (current) and the estimated
management targets calculated based on the relative target biomass (B
target), the SPR target, and the maximum sustainable yield (MSY).

The 2023 spawning biomass relative to unfished equilibrium spawning
biomass, based on the 2022 fishing year, is 80.5397\%, above the
management target of 40\% of unfished spawning output. The relative
biomass and the ratio of the estimated SPR to the management target
(\(\text{SPR}_{50\%}\)) across all model years are shown in Figure
\ref{fig:phase} where warmer colors (red) represent early years and
colder colors (blue) represent recent years. There have been periods
where the stock status has decreased below the target and limit relative
biomass, and fishing intensity has been higher than the target fishing
intensity based on \(\text{SPR}_{50\%}\).

\subsection{Management performance}\label{management-performance-2}

Rougheye/Blackspotted rockfishes removals have been below the equivalent
Annual Catch Limit (ACL) over the recent decade (Table
\ref{tab:manage}). The ACL declined in 2017 relative to earlier years
based on the 2015 assessment of Rougheye/Blackspotted rockfishes (Cope
et al. 2016). In the last ten years, catches peaked in 2016 at 369 mt.
Since then catches have declined to a recent low of 130 mt in 2020 with
the catches in the final two model years remaining low with 197 mt in
2021 and 166 mt in 2022. The OFL has not been exceeded in any year over
the past 10 years.

\subsection{Harvest Projections and Decision
Tables}\label{harvest-projections-and-decision-tables-1}

The Rougheye/Blackspotted rockfishes assessment is being considered as a
category 1 assessment with a \(P^*\) = 0.45, \(\sigma\) = 0.50, and a
time-varying buffer applied to set the ABC below the OFL. These
multipliers are also combined with the rockfish MSY proxy of
SPR\textsubscript{50} and the 40-10 harvest control rule to calculate
OFLs and ACLs. A twelve-year (2023-2034) projection of the reference
model using these specifications along with input removals for 2023 and
2024 provided by the Groundfish Management Team (Katie Pierson, ODFW,
pers. comm.) is provided in Table \ref{tab:project}.

Uncertainty in management quantities for the reference model was
characterized by exploring various model specifications in a decision
table, with the desire for states of nature to represent uncertainty in
both scale and relative stock status Initial explorations considering
alternative specifications of natural mortality. This was based on using
the estimated \(M\) scenario as a low state of nature and applying the
sex-specific \(M\) values from the 2023 Oregon model as the high state
of nature. These produced wide states of nature (Figure
\ref{fig:stateofnature_SO} and Figure \ref{fig:stateofnature_depl}).
Discussion with the STAR panel led to defining two other states of
nature based on the reference model uncertainty in ending spawning
output. Low and high states of nature were determined by applying an
initial recruitment (\(lnR_0\)) value that lead to current spawning
output values equivalent to the 12.5\% and 87.5\% percentile values from
the current spawning output distribution (Figure
\ref{fig:stateofnature_SO} and Figure \ref{fig:stateofnature_depl}) that
are not as widely spread as the initial states of nature, but are
constructed from the current model specifications. The resultant
decision table (Table \ref{tab:dec-tab}) was built around the initial
\(lnR_0\) states of nature approach. The catch rows assume P* values of
0.45 and 0.4, then a constant catch using the yield at FSPR=0.5.

\subsection{Evaluation of Scientific
Uncertainty}\label{evaluation-of-scientific-uncertainty-1}

\#The model-estimated uncertainty around the 2027 spawning biomass was
\(\sigma\) = 100 and the uncertainty around the OFL was \(\sigma\) = NA.
This is likely underestimate of overall uncertainty because of the
necessity to fix some life history parameters such as natural mortality
and steepness, as well as a lack of explicit incorporation of model
structural uncertainty. The alternative states of nature used to bracket
uncertainty in the decision table assist with encapsulating model
structure uncertainty.

\subsection{Research and Data Needs}\label{research-and-data-needs-1}

This section briefly highlights progress on research and data needs
identified in the most recent (2015) Rougheye/Blackspotted rockfishes
assessment, and then provides recommendations for future research.

Research and data needs identified in the last assessment (italics) are
listed here followed by a brief response for each.

\newpage{}

\subsection{Acknowledgements}\label{sec-acknowledgements}

\newpage{}

\subsection{References}\label{references}

\newpage{}

\subsection{Tables}\label{tables}

\newpage{}

\subsection{Figures}\label{figures}

\subsubsection{Introduction}\label{introduction-1}

\begin{figure}

\centering{

\includegraphics{SAR_USWC_Rougheye_and_Blackspotted_Rockfishes_skeleton_files/figure-pdf/fig-map-1.pdf}

}

\caption{\label{fig-map}Map of the assessment area.}

\end{figure}%

\begin{figure}

\centering{

\includegraphics{plots_4_doc/SB_comps.png}

}

\caption{\label{fig-SO_comp}Estimates of spawning biomass (current
spawning output/unfished spawning output) for the Rougheye/Blackspotted
rockfish complex from the two most recent Alaska (Bering Sea/Aleutian
Islands (BSAI) and Gulf of Alaska (GOA)) and the 2013 U.S. west coast
stock assessment.}

\end{figure}%

\begin{figure}

\centering{

\includegraphics{plots_4_doc/Status_comp.png}

}

\caption{\label{fig-RSS_comp}Estimates of relative stock size (current
spawning output/unfished spawning output) relative to 1977 (the common
year in all stock assessments compared) for the Rougheye/Blackspotted
rockfish complex from the two most recent Alaska (Bering Sea/Aleutian
Islands (BSAI) and Gulf of Alaska (GOA)) and the 2013 U.S. west coast
stock assessment.}

\end{figure}%

\begin{figure}

\centering{

\includegraphics{ref_model/plots/data_plot.png}

}

\caption{\label{fig-data}Data used in the base model.}

\end{figure}%

\newpage{}

\subsection{Notes}\label{notes}

\newpage{}

\subsection*{Appendices}\label{sec-appendix}
\addcontentsline{toc}{subsection}{Appendices}

\phantomsection\label{refs}
\begin{CSLReferences}{1}{0}
\bibitem[\citeproctext]{ref-cope_assessments_2016}
Cope, J. M., David B. Sampson, Andi Stephens, Meisha Key, Patrick P.
Mirick, Megan M. Stachura, Tien-Shui Tsou, et al. 2016. {``Assessments
of {California}, {Oregon}, and {Washington} Stocks of Black Rockfish
(\emph{{Sebastes} Melanops}) in 2015.''} Pacific Fishery Management
Council, 7700 Ambassador Place NE, Suite 200, Portland, OR 97220:
Pacific Fishery Management Council.

\bibitem[\citeproctext]{ref-francis_data_2011}
Francis, R. I. C. Chris. 2011. {``Data Weighting in Statistical
Fisheries Stock Assessment Models.''} \emph{Canadian Journal of
Fisheries and Aquatic Sciences} 68 (6): 1124--38.
\url{https://doi.org/10.1139/f2011-025}.

\bibitem[\citeproctext]{ref-mcallister_bayesian_1997}
McAllister, M. K., and J. N. Ianelli. 1997. {``Bayesian Stock Assessment
Using Catch-Age Data and the Sampling --- Importance Resampling
Algorithm.''} \emph{Canadian Journal of Fisheries and Aquatic Sciences}
54 (2): 284--300. \url{https://doi.org/10.1139/f96-285}.

\bibitem[\citeproctext]{ref-thorson_model-based_2017}
Thorson, James T., Kelli F. Johnson, R. D. Methot, and I. G. Taylor.
2017. {``Model-Based Estimates of Effective Sample Size in Stock
Assessment Models Using the {Dirichlet}-Multinomial Distribution.''}
\emph{Fisheries Research} 192: 84--93.
\url{https://doi.org/10.1016/j.fishres.2016.06.005}.

\end{CSLReferences}




\end{document}
